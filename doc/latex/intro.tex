


%%%%%%%%%%%%%%%%%%%%%%%%%%%%%%%%%%%%%%%%%%%%%%%%%%%%%%%%%%%%%%%%%%%%%%%%
\chapter{Introduction}
%%%%%%%%%%%%%%%%%%%%%%%%%%%%%%%%%%%%%%%%%%%%%%%%%%%%%%%%%%%%%%%%%%%%%%%%


\SIM is a heterogeneous architecture simulator, which is trace-driven
and cycle-level. It thoroughly models architectural behaviors,
including detailed pipeline stages, multi-threading, and memory
systems. Currently, \SIM support x86 and NVIDIA PTX instruction set
architectures (ISA). \SIM is capable of simulating a variety of
architecreus, such as Intel's Sandy Bridge~\cite{sandybridge} and
NVIDIA's Fermi~\cite{fermi}.  It can simulate homogeneous ISA
multicore simulations as well as heterogeneous ISA multicore
simulations.

MacSim is a microarchitecture simulator that simulates detailed
pipeline stages (in-order and out-of-order) and the memory system components 
including caches, NoC, and memory controllers. It supports asymmetric 
multicore configurations as well as SMT or MT architectures.

Currently interconnection network model (based on IRIS) and power
model (based on McPat~\cite{mcpat}) are implemented. ARM ISA support is
on-progress. MacSim is also one of the components of SST~\cite{sst} so
multiple MacSim simulators can run concurrently.





%%%%%%%%%%%%%%%%%%%%%%%%%%%%%%%%%%%%%%%%%%%%%%%%%%%%%%%%%%%%%%%%%%%%%%%%
\section*{Macsim version information}
%%%%%%%%%%%%%%%%%%%%%%%%%%%%%%%%%%%%%%%%%%%%%%%%%%%%%%%%%%%%%%%%%%%%%%%%


\begingroup
\renewcommand\descriptionlabel[1]{\textit{\hspace\labelsep{#1}}}
%\renewcommand\descriptionlabel[1]{\hspace\labelsep\cs{#1}}
\begin{description}\firmlist
\item[2.2.0-  September 2015] 
  \Verb+macsim/tags/macsim-2.2.0/+
\item[2.1.1 - May 2015] 
  \Verb+macsim/tags/macsim-2.1.1/+
\item[2.1.0 - May 2015] 
  \Verb+macsim/tags/macsim-2.1.0/+
\item[2.0.4 - October, 2014] 
  \Verb+macsim/tags/macsim-2.0.4/+
\item[2.0.3 - August, 2014] 
  \Verb+macsim/tags/macsim-2.0.3/+
\item[2.0.2 - April, 2014] 
  \Verb+macsim/tags/macsim-2.0.2/+
\item[2.0 - September, 2013] 
  \Verb+macsim/tags/macsim-2.0/+
\item[1.2.1 - April, 2013] 
  \Verb+macsim/tags/macsim-1.2.1/+
\item[1.2 - October, 2012] 
  \Verb+macsim/tags/macsim_1.2+ 
\item[1.1 - October, 2012] 
  \Verb+macsim/tags/macsim_1.1+ 
\item[1.0 - February, 2012] Initial release
  \Verb+macsim/tags/macsim_1.0+ 
\end{description}
\endgroup


%%%%%%%%%%%%%%%%%%%%%%%%%%%%%%%%%%%%%%%%%%%%%%%%%%%%%%%%%%%%%%%%%%%%%%%%
\section*{Macsim Code Contributors}
%%%%%%%%%%%%%%%%%%%%%%%%%%%%%%%%%%%%%%%%%%%%%%%%%%%%%%%%%%%%%%%%%%%%%%%%
The MacSim code is mainly developed at the HpArch research group. The
main source code contributors are as follows: Hyesoon Kim, Jaekyu Lee,  Nagesh B. Lakshminarayana, Jaewoong Sim,
Jieun Lim, Tri Pho, Hyojong Kim, Ramyad Hadidi.  This list keeps 
increasing as we add more features. 

%%%%%%%%%%%%%%%%%%%%%%%%%%%%%%%%%%%%%%%%%%%%%%%%%%%%%%%%%%%%%%%%%%%%%%%%
\section*{Macsim  Mailing List}
%%%%%%%%%%%%%%%%%%%%%%%%%%%%%%%%%%%%%%%%%%%%%%%%%%%%%%%%%%%%%%%%%%%%%%%%

Please join macsim-dev@googlegroups.com for macsim mailing list. 
Internal developers use macsim-dev-hparch@googlegroups.com. If you
have  questions about joining the mailing list, please send email to hyesoon@cc.gatech.edu 

















% LocalWords:  Macsim MacSim NVIDIA PTX multicore RISC uops microarchitecture
% LocalWords:  NoC SMT McPat architecreus sandybridge begingroup renewcommand
% LocalWords:  descriptionlabel textit hspace labelsep cs firmlist endgroup
