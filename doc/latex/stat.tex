

%%%%%%%%%%%%%%%%%%%%%%%%%%%%%%%%%%%%%%%%%%%%%%%%%%%%%%%%%%%%%%%%%%%%%%%%
\chapter{Simulation Statistics}
\label{sec:stat}
%%%%%%%%%%%%%%%%%%%%%%%%%%%%%%%%%%%%%%%%%%%%%%%%%%%%%%%%%%%%%%%%%%%%%%%%

A simple framework for collecting statistics (hereafter referred to as stats)
  during simulation is provided.  Stats can be either global (includes
      data from all cores) or per core.

%%%%%%%%%%%%%%%%%%%%%%%%%%%%%%%%%%%%%%%%%%%%%%%%%%%%%%%%%%%%%%%%%%%%%%%%
\section{Stat Types}
%%%%%%%%%%%%%%%%%%%%%%%%%%%%%%%%%%%%%%%%%%%%%%%%%%%%%%%%%%%%%%%%%%%%%%%%

The following stat types are supported:

\begin{description}

  \item [COUNT] for counting the number of occurances of an event. Eg. number
  of cache hits. 

  \item [RATIO] for calculating the ratio of number of occurances of one event
  over another. Eg. (number of cache hits / number of cache accesses) i.e.,
  cache hit ratio.

  \item [DIST]  for calculating the proportion of each event in a group of
  events.  \ignore{for defining a group of related events and calculating the
    number of occurances of each event in the group as a percent of the sum of
      the number of occurances of all events in the group.}  Eg. If the user
      wants to know what percent of L1 data cache accesses (in a 2-level
          hierarchy) resulted in L1 hits, L2 hits or memory accesses, then the
      user should define a distribution consisting on three events - L1 hits,
      L2 hits and L2 misses  - and update the counter for each event correctly. 

\end{description}


Note that a simulation will output two values for each stat. One is the
raw value i.e. the number of occurances of the event associated with the
stat and the other value is the value calculated based on the type
of the stat.


%%%%%%%%%%%%%%%%%%%%%%%%%%%%%%%%%%%%%%%%%%%%%%%%%%%%%%%%%%%%%%%%%%%%%%%%
\section{Adding a new stat}
%%%%%%%%%%%%%%%%%%%%%%%%%%%%%%%%%%%%%%%%%%%%%%%%%%%%%%%%%%%%%%%%%%%%%%%%

New stats can be defined by adding DEF\_STAT statements to any of the
\textit{*.stat.def} files in the \textit{trunk/def/} directory or by creating a
\textit{.stat.def} file including the definitions in the \textit{trunk/def/}
directory.  To define a per core statistic specify PER\_CORE at the
end of each DEF\_STAT statement below.



\begin{description}

\item[COUNT Stat:] \Verb+ +

\Verb+DEF_STAT(STAT_NAME, COUNT, NO_RATIO [, PER_CORE])+

Eg: 
\begin{Verbatim}
DEF_STAT(INST_COUNT_TOT, COUNT, NO_RATIO)
DEF_STAT(INST_COUNT, COUNT, NO_RATIO, PER_CORE)
\end{Verbatim}

\item[RATIO Stat:] \Verb+ +

\Verb+DEF_STAT(STAT_NAME, RATIO, BASE_STAT_NAME [, PER_CORE])+


In addition to defining the RATIO stat itself, a base stat of type COUNT has to
be defined as well. The value of the base stat is used as the denominator in
calculating the ratio. 

Eg: 
\begin{Verbatim}
DEF_STAT(DISPATCHED_INST, COUNT, NO_RATIO)
DEF_STAT(DISPATCH_WAIT, RATIO, DISPATCHED_INST)
\end{Verbatim}

\item[DIST Stat:] \Verb+ +

\begin{Verbatim}
DEF_STAT(STAT_NAME_START, DIST, NO_RATIO [, PER_CORE])
DEF_STAT(STAT_NAME, COUNT, NO_RATIO [, PER_CORE])*
DEF_STAT(STAT_NAME_END, DIST, NO_RATIO [, PER_CORE])
\end{Verbatim}

The definition of a DIST stat requires at least two stats.

Eg: 
\begin{Verbatim}
DEF_STAT(SCHED_FAILED_REASON_SUCCESS, DIST, NO_RATIO, PER_CORE)
DEF_STAT(SCHED_FAILED_OPERANDS_NOT_READY, COUNT, NO_RATIO, PER_CORE)
DEF_STAT(SCHED_FAILED_NO_PORTS, DIST, NO_RATIO, PER_CORE)
\end{Verbatim}

\end{description}


%%%%%%%%%%%%%%%%%%%%%%%%%%%%%%%%%%%%%%%%%%%%%%%%%%%%%%%%%%%%%%%%%%%%%%%%
\section{Updating Stats}
%%%%%%%%%%%%%%%%%%%%%%%%%%%%%%%%%%%%%%%%%%%%%%%%%%%%%%%%%%%%%%%%%%%%%%%%


Macros are provided to update the value of stats. STAT\_EVENT and
STAT\_EVENT\_M increment and decrement the value of a global stat by 1 and take
the name of the stat to be updated as their argument.  STAT\_EVENT\_N is used
to increment the value of a global stat by more than than 1. It takes the name
of the stat to be incremented and the value to be added as its arguments.
STAT\_CORE\_EVENT and STAT\_CORE\_EVENT\_M increment and decrement the value of
a per core stat by 1. These take core id and the name of the stat to be
incremented/decremented as their parameters. For example,

\begin{Verbatim}
STAT_EVENT(INST_COUNT_TOT); // increments global stat INST_COUNT_STAT by 1
STAT_EVENT_N(INST_COUNT_TOT, 2); // increments global stat INST_COUNT_STAT by 2
STAT_EVENT_M(INST_COUNT_TOT); // decrements global stat INST_COUNT_STAT by 1
STAT_CORE_EVENT(0, INST_COUNT); // increments stat INST_COUNT for core 0 by 1
\end{Verbatim}



%%%%%%%%%%%%%%%%%%%%%%%%%%%%%%%%%%%%%%%%%%%%%%%%%%%%%%%%%%%%%%%%%%%%%%%%
\section{Simulation output}
%%%%%%%%%%%%%%%%%%%%%%%%%%%%%%%%%%%%%%%%%%%%%%%%%%%%%%%%%%%%%%%%%%%%%%%%

At the end of a simulation several files with the extension stat.out are
generated, these files include the stat values at the end of the
simulation. As mentioned, for each stat two values are generated, one is the
raw stat value and other is a value calculated based on the type of the
stat. For simulations with multiple applications, multiple sets of stat files
are generated. Each simulated application is assigned an integer id (these ids
    are assigned according to the order in which the applications appear in the
    trace\_file\_list), when an application terminates (for the first time,
      note that applications may be repeated), stat files suffixed with the the
    id of the application, i.e.  *.stat.out.<appl\_id>, are generated. These
    stat files contain the value of the stats until that point in the
    simulation. At the end of the simulation, *.stat.out files are generated as
    usual.



%%%%%%%%%%%%%%%%%%%%%%%%%%%%%%%%%%%%%%%%%%%%%%%%%%%%%%%%%%%%%%%%%%%%%%%%
\section{Important Stats}
%%%%%%%%%%%%%%%%%%%%%%%%%%%%%%%%%%%%%%%%%%%%%%%%%%%%%%%%%%%%%%%%%%%%%%%%

\begin{table}[htb]
\begin{footnotesize}
\begin{center}
\caption{Some important Stats} 
\label{table:stats}
\begin{tabular}{|l||l|c|l|}
\hline 
INST\_COUNT\_TOT            & \# of instructions                                    &      & general.stat.out \\ \hline 
INST\_COUNT\_CORE\_[0-11]   & \# of instructions in only the specificed core [0-11] & core & general.stat.out \\ \hline 
CYC\_COUNT\_TOT             & simulated cycles                                      &      & general.stat.out \\ \hline 
CYC\_COUNT\_CORE\_[0-11]    & simulated cycles in only the specificed core [0-11]   &      & general.stat.out \\ \hline 
CYC\_COUNT\_X86             & simulated cycles for x86 only                         &      & general.stat.out \\ \hline 
CYC\_COUNT\_PTX             & simulated cycles for ptx only                         &      & general.stat.out \\ \hline \hline 
                            & \# of fp instructions                                 &      &                  \\ \hline 
                            & \# of int instructions                                &      &                  \\ \hline 
                            & \# of load instructions                               &      &                  \\ \hline  
                            & \# of store instructions                              &      &                  \\ \hline  
BP\_ON\_PATH\_CORRECT       & \# of correctly predicted branches (DIST)             & core & bp.stat.def      \\ \hline  
BP\_ON\_PATH\_MISPREDICT    & \# of mis-predicted branches (DIST)                   & core & bp.stat.def      \\ \hline  
BP\_ON\_PATH\_MISFETCHT     & \# of mis-fetch branches (BTB MISS)(DIST)             & core & bp.stat.def      \\ \hline  
ICACHE\_HIT, ICACHE\_MISS   & \# of I-cache hitt,miss (DIST)                        &      & memory.stat.def  \\ \hline  
L[1-3]\_HIT\_CPU            & \# of l[1-3]cache hits from CPU                       &      & memory.stat.def  \\ \hline 
L[1-3]\_HIT\_GPU            & \# of l[1-3]cache hits from GPU                       &      & memory.stat.def  \\ \hline 
L[1-3]\_MISS\_CPU           & \# of l[1-3]cache misses from CPU                     &      & memory.stat.def  \\ \hline 
L[1-3]\_MISS\_GPU           & \# of l[1-3]cache misses from GPU                     &      & memory.stat.def  \\ \hline  \hline 
AVG\_MEMORY\_LATENCY        & average memory latency                                &      & memory.stat.def  \\ \hline \hline 
TOTAL\_DRAM                 & \# of DRAM accesses                                   &      & memory.stat.def  \\ \hline  
TOTAL\_DRAM\_READ           & \# of DRAM reads                                      &      & memory.stat.def  \\ \hline  
TOTAL\_DRAM\_WB             & \# of DRAM write backs                                &      & memory.stat.def  \\ \hline  
                            & \# of register reads                                  &      &                  \\ \hline  
                            & \# of register writes                                 &      &                  \\ \hline   \hline 
COAL\_INST, UNCOAL\_INST    & coalesced/uncoalesced mem requests (DIST)             &      & memory.stat.def  \\ \hline 




\end{tabular}
\end{center}
\end{footnotesize}
\end{table} 


