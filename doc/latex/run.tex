
\clearpage
\section{Macsim Installation and Run}

\subsection{Download}

\SIM source code is maintained using the subversion. 
You can check out the \SIM copy by

\smallskip
\begin{lstlisting}
svn co https://svn.research.cc.gatech.edu/macsim/trunk macsim-readonly --username readonly
\end{lstlisting}
\smallskip


\subsection{Wiki and Other Supports}

We manage the google project page in the following url:

\smallskip
\begin{lstlisting}
http://code.google.com/p/macsim/
\end{lstlisting}
\smallskip


\subsection{Build Requirement}

\SIM requires following to build properly.

\begin{description}

  \item[Operating System] Currently, we only support linux
  distributions. Tested systems are as follows:

  \smallskip
  \begin{lstlisting}
  Ubuntu
  Redhat (todo)
  \end{lstlisting}
  \smallskip

  \item[Compiler] Any compiler that can supports the C++0x (or C++11)
  standard library. Currently, we tested follwing compilers:
        
  \smallskip 
  \begin{lstlisting}
  gcc   
  icc (todo)
  \end{lstlisting} 
  \smallskip

  \item[Autotools] - You need to have autotools (automake, autoconf,
  ...) version 2.65 or higher. You can install autotools by

  \smallskip
  \begin{lstlisting}
  Ubuntu: apt-get install autotools-dev automake autoconf
  Redhat: todo
  \end{lstlisting}
  \smallskip


\end{description}

\ignore{
\subsection{Directory Structure}

This section explains the directory structure of \SIM simulator.

\smallskip
\begin{lstlisting}
macsim/
  tag/ branch/ trunk/
\end{lstlisting}
\smallskip

\textit{Tag} directory has tagged version of \SIM
simulators. \textit{Branch} directory is for diverged \SIM, which is
currently empty. \textit{Trunk} directory is current working directory
for \SIM.

Following is more detailed information about \textit{Trunk} directory.

\smallskip
\begin{lstlisting}
trunk/
  bin/ def/ doc/ params/ scripts/ src/ tools/
\end{lstlisting}
\smallskip

\textit{Bin} directory contains the \SIM binary after the building
process. \textit{def} directory has knob (Section~\ref{sec:knob}) and
statistics (Section~\ref{sec:stat}) definitions. \textit{doc} has the
documentation. \textit{scripts} includes several scripts files that
are using during the building process. \textit{src} contains all
source files. \textit{tools} has several useful tools.
\ignore{ (Section~\ref{sec:tool}) }
}





\subsection{Installation}

The GNU Autotools (automake, autoconf) have been used for building
\SIM simulator. After initial check out of the \SIM copy, following commands are necessary.

\smallskip
\begin{lstlisting}
aclocal 
automake 
--add-missing 
autoconf 
./configure 
make
\end{lstlisting}
\smallskip

You can combine above commands in a line:

\smallskip
\begin{lstlisting}
aclocal && automake --add-missing && autoconf && ./configure && make
\end{lstlisting}
\smallskip

We provide autogen.sh script file to simplify the building process.

\smallskip
\begin{lstlisting}
./autogen.sh
make
\end{lstlisting}
\smallskip

The binary \textit{macsim} will be generated in the
\textit{trunk/bin/} directory.





\subsection{Build Types}

We provide three different build types.

\begin{itemize}
  \item opt : default, optimized version (-O3 flag)
  \item dbg : debug version (-g3 flag)
  \item gpf : gprof version (-pg flag)
\end{itemize}

To build a certain type, you need to specify the option
after \textit{make} command. For example,

\smallskip
\begin{lstlisting}
make opt
make dbg
make gpf
\end{lstlisting}
\smallskip





\subsection{How To Run \SIM}

To run \textit{macsim} binary, two additional files are required in
the same directory.

\begin{itemize}
  \item params.in - defines architectural parameters that will
  overwrite the default value.

  \item trace\_file\_list - defines the number of traces to run and
  the path of each trace
\end{itemize}

\subsubsection{params.in}

We provide sample parameter files for various architectures (Intel
CPUs, NVIDIA GPUs, ...) in
the \textit{macsim-top/trunk/params}. Following is sample content of
params.in file.

\smallskip
\begin{lstlisting}
# Simulation Configuration
num_sim_cores 1
num_sim_small_cores 0
num_sim_medium_cores 0
num_sim_large_cores 1
core_type ptx
large_core_type x86
cpu_frequency 4
gpu_frequency 1.5
sim_cycle_count 0
max_insts 500000000
heartbeat_interval 1000000
forward_progress_limit 50000


# Common Core Configuration
fetch_policy rr
mt_no_fetch_br 1
one_cycle_exec 0
\end{lstlisting}
\smallskip

The first literal is the name of parameter and the second literal is a
value that will overwrite the default value. Section~\ref{sec:knob}
details how to add, modify, and use these parameters.


\subsubsection{trace\_file\_list}

Following is the content of a sample trace\_file\_list.

\smallskip
\begin{lstlisting}
2
/trace/ptx/cuda2.2/FastWalshTransform/kernel_config.txt
/trace/ptx/cuda2.2/BlackScholes/kernel_config.txt
\end{lstlisting}
\smallskip

The first line is the number of traces that \SIM will run and
following lines are the path of each traces. Each top level trace
configuration file contains various information. Following is the
content of a top-level trace information file.


\smallskip
\begin{lstlisting}
1 x86
0 0
\end{lstlisting}
\smallskip

In the first line, the first number indicates the number of threads in
the application and second literal specifies the type of the
application. Trailing lines are the information of each thread
({thread id}, {thread starting point in terms of the instruction count
of the main thread (thread 0)}). Section~\ref{sec:traceformat}
describes the detail of the trace file format.



\subsubsection{Run}

The following is the sample command lines to run \SIM simulator.

\smallskip
\begin{lstlisting}
./macsim [additional commands]
./macsim --l2_assoc=8 --stdout=stdout --stderr=stderr
./macsim --l1_assoc=4 
\end{lstlisting}
\smallskip




% LocalWords:  Macsim svn macsim readonly username params src Autotools aclocal
% LocalWords:  automake autoconf autogen dbg gpf gprof CPUs NVIDIA GPUs num sim
% LocalWords:  ptx cpu gpu insts rr br Wiki google url linux Redhat todo gcc
% LocalWords:  icc autotools dev stdout stderr
