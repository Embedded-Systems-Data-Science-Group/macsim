
\clearpage
\section{The Pipeline Stages}

This section describes the basic feature of each pipeline stage. 
Table~\ref{table:pipeline} shows the list of files that are relevant to each pipeline stage. 
Please see the doxygen file to get the up-to-date information. 


\begin{table}[htb]
\begin{tabular}{|c||c|}
\hline 

pipeline stage & file names \\ \hline \hline 
main simulator & main.cc, core.cc  \\ \hline 

fetch stage & frontend.cc, fetch\_factor.cc bp.cc bp\_*.cc   \\ \hline 
decode stage &  allocate.cc \\ \hline 
allocate stage & allocate.cc allocate\_*.cc rob.cc rob\_*.cc map.cc  \\ \hline 
schedule stage & schedule.cc schedule\_*.cc  \\ \hline 
execution stage & exec.cc \\ \hline 
retire stage & retire.cc \\ \hline 
memory system &  port.cc cache.cc *\_cache.cc dram.cc memory.cc \\ 

other supporting files &  statistics.cc bug\_detector.cc   \\  \hline \hline 

process manager & process\_manager.cc \\ \hline 
\end{tabular}
\caption{Pipeline stage and corresponding source code names}
\label{table:pipeline}
\end{table}



\subsection{Fetch Stage}
This pipeline stage models the instruction fetch stage. 

The main tasks of this stage are 
\begin{enumerate}
\item Determine a thread from which instructions are fetched 
\item Check whether the thread can actually fetch a new instruction or not  
\item access I-cache  (when icache miss, the front-end cannot fetch an instruction) 
\item read a uop from trace from (call {\tt get\_uops\_from\_traces})
\item BTB access and branch prediction 
\item send uop to {\tt q\_fe} to model the depth of front-end 
\end{enumerate}
The depth of front-end stage is set by {\tt fetch\_latency} and it basically sets the length of 
{\tt q\_fe  }

Different fetch polices are implemented using virtual function. {\tt fetch\_factory} function sets 
different {\tt frontend\_c}. Currently, a thread is selected based on the round-robin policy. If the next thread is blocked to fetch, the next candidate is selected based on the round-robin policy. 

 Different branch predictor policies are also implemented and it is also set by {\tt bp\_factory}. The type of branch predictor or the fetch polices are set by knobs. 



\subsubsection{Trace Read} 
Trace read is mainly performed in trace\_read.cc. 
The main task of trace read is reading an macro instruction from trace file and convert it to uops. 
A simple decoding algorithm is used to generate micro uops. 
To improve the simulation speed, decoded uops are stored in a hash table. 
Which trace file to open is decided in process\_manager.cc 



\subsection{Decode and  Allocate Stage }
This stage is simply implemented with {\tt  allocate queue}.
At the end of this stage,  the simulator performs resource allocations such as rob entry and load-store buffer. 
Branch target is resolved in this stage as well. 

\subsection{Schedule Stage}
The scheduler stage implements scheduling logic. Currently it supports in-order, out-of-order, and GPU scheduling mechanism. 
First, it checks the source code availability and then functional unit availability-by checking ports. 
 The scheduler is implemented in a virtual function so different scheduling policies overload the scheduling functions. 

\subsection{Execution Stage}

In this stage, both memory instructions and other instructions are executed. 
\begin{enumerate} 

\item The latency of uop is checked in this stage. 
{\tt ../def/uoplatency\_ptx.def} includes the latency of uops. 
{\tt uop->m\_done\_cycle} indicates the end of execution time. 

\item This stage also controls the number of function units by using the following three parameters. 
{\tt int\_sched\_rate, mem\_sched\_rate, fp\_sched\_rate}

\item For memory instructions, the simulator checks dcache at the beginning of the execution cycle 
and if there is a dcache miss, it is handled separately. Memory system is explained in Section~\ref{sec:memory}

\item This stages also handles uncoalesced memory operations. Uncoalesced memory requests are handed 
by using children uop structures. In {\tt trace\_read.cc} file already generated children uop when it detects uncoalesced memory requests. 

\item Branch instructions are resolved in this stage as well. 
\end{enumerate} 

\subsection{Retire Stage}
This stage handles retiring uops. It supports in-order retirement. 
This stage also frees uop structures. 

% LocalWords:  doxygen frontend bp uops  GPU uop dcache uncoalesced BTB fe mem
% LocalWords:  sched fp icache ptx
