
\clearpage
\section{The GPU Model}


We model our GPU cores similar to NVIDIA Fermi~\cite{fermi}.
%Figure~\ref{fig:g80} shows an overview of the GPU architecture that we modeled in our simulator. 
The GPU architecture consists of a scalable number of {\em streaming
multiprocessors} (SMs), each containing eight {\em streaming processor}
(SP) cores, two special function units (SFUs), a multithreaded
instruction fetch and issue unit, a read-only constant cache, and a
16KB read/write shared memory~\cite{lin:nic08}.
In our simulator SMs are names as cores and SPs are just one of the functional units. 


\ignore{
\begin{figure}[htb]
\centering
\psfig{file=figs/g80, angle=0, width=\figwidth}
\caption{An overview of the GPU architecture}
\label{fig:g80}
\end{figure}
}

The SM executes a batch of 32 threads together called a {\em
warp}. In the simulator, one warp is treated as one thread. 
The trace generator already formed a warp and stored them. 

\subsection{Changes to support GPU}
\subsubsection{Front-end changes}
The following components are modified to support GPUs. 
\begin{enumerate}
\item Handling divergent branch
\item Fetch scheduling (when to block a thread)
\end{enumerate}
\subsubsection{Scheduler}
Different scheduling policy can be implemented. The default is the round-robin policy. 

\subsubsection{Process Manager}

\begin{enumerate}
\item Block Dispatch: Threads are dispatched as a block granularity. 
The simulator waits until all threads within a block are finished 
before it dispatches another block. 
\item Occupancy: When a trace is generated from Ocelot, it also emits the number of required 
physical registers and memory usages. So the simulator uses that information to decide 
how many blocks can be allocated for each core. This value can be overwritten by a KNOB as well. 
\end{enumerate}

\subsubsection{Execution Stage}
Since one warp is treated as a functional unit, execution stage itself does not require too much change. Major changes are in handling uncoalesced memory  requests. 

\subsubsection{Memory system}
The simulator includes read-only cache, texture cache and constant cache to model GPUs. 

\ignore{



Executing a warp instruction applies the instruction to 32
threads, similar to executing a SIMD instruction like an SSE
instruction~\cite{sse} in X86. However, unlike SIMD instructions, the
concept of warp is not exposed to the programmers, rather programmers
write a program for one thread, and then specify the number of
parallel threads in a block, and the number of blocks in a kernel
grid. The Tesla architecture forms a warp using a batch of 32
threads~\cite{cuda:course, sc2008_cuda} and in the rest of the paper
we also use a warp as a batch of 32 threads.\ignore{\footnote{The CUDA
manual~\cite{cuda_manual} describes a half warp (16 threads) as a
minimum unit of execution. However, to simplify the model, we use only
warp as a minimum unit of execution.}}

All the threads in one block are executed on one SM together. One SM
can also have multiple concurrently running blocks. The number of
blocks that are running on one SM is determined by the resource
requirements of each block such as the number of registers and shared
memory usage. The blocks that are running on one SM at a given time
are called {\em active blocks} in this paper. Since one block
typically has several warps (the number of warps is the same as the
number of threads in a block divided by 32), the total number of
active warps per SM is equal to the number of warps per block times
the number of active blocks.

The shared memory is implemented within each SM multiprocessor as an
SRAM and the global memory is part of the offchip DRAM. The shared
memory has very low access latency (almost the same as that of
register) and high bandwidth. However, since a warp of 32 threads
access the shared memory together, when there is a bank conflict
within a warp, accessing the shared memory takes multiple cycles.

}


% LocalWords:  GPU NVIDIA SMs SFUs multithreaded SPs GPUs uncoalesced SIMD CUDA
% LocalWords:  SRAM offchip
