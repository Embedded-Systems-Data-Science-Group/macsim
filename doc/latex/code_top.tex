
\chapter{Overview of MacSim Code Structures}
\label{sec:codetop}

This section provides a brief overview of the organization of the code in
MacSim. Code in MacSim is organized into many classes, for a descpription of
the classes please refer to the doxygen documentation.

  \ignore{
For each simulation, an instance of \textit{macsim\_c} is created.  In
\textit{macsim\_c}, there is an instance of \textit{process\_s} for each
application being simulated. An application can contain multiple threads each
represented by a \textit{thread\_s} structure. In case of GPU applications, an
instance of \textit{thread\_s} represents a warp. The threads of an application
can run on multiple cores each of which is an \textit{core\_c} object. For GPU
applications, each core represents an SM. Each core has a member variable for
each stage of the pipeline. The classes for the different pipeline stages are:
\textit{frontend\_c, allocate\_c/smc\_allocate\_c,
  schedule\_io\_c/schedule\_ooo\_c/schedule\_smc\_c, exec\_c, retire\_c}. A
  \textit{core\_c} object also contains objects of \textit{rob\_c/smc\_rob\_c}
  (ROB), \textit{cache\_c} (I-cache), \textit{readonly\_cache\_c} (Const cache
      and Texture cache), \textit{sw\_managed\_cache\_c} (Shared Memory),
  \textit{bp\_data\_c} (Branch predictor), \textit{hwp\_common\_c} (Hardware
      Prefetcher). Note that some of these components are valid for CPUs only,
  while some others are valid for GPUs only.  According to the speficied memory
  hierarchy, an instance of a class that derives from \textit{memory\_c} is
  created. This object is a member of \textit{macsim\_c} and is accessed by all
  cores in the simulation. The \textit{memory\_c} object includes objects L1,
  L2 and L3 caches and also DRAM controllers which are objects of classes that
  derive from \textit{dram\_controller\_c}.
  }


For each simulation, an instance of \textit{macsim\_c} is created.  In
\textit{macsim\_c}, there is an instance of \textit{process\_s} for each
application being simulated. An application can contain multiple threads each
represented by a \textit{thread\_s} structure. In case of GPU applications, an
instance of \textit{thread\_s} represents a warp. The threads of an application
can run on multiple cores each of which is an \textit{core\_c} object. For GPU
applications, each core represents an SM. Each core has a member variable for
each stage of the pipeline.  A \textit{core\_c} object also contains objects
for hardware structures such as ROB, I-cache, Shared Memory, branch predictor
and so on. Note that some of these components are valid for CPUs only, while
some others are valid for GPUs only.  According to the speficied memory
hierarchy, an instance of a class that derives from \textit{memory\_c} is
created. This object is a member of \textit{macsim\_c} and is accessed by all
cores in the simulation. The \textit{memory\_c} object includes objects for L1
D-cache, L2 and L3 caches and also DRAM controllers which are objects of
classes that derive from \textit{dram\_controller\_c}. 


\section{Simulator Execution}

All pipeline stages, units such as cache and memory controller, components such
as memory system and core implement the \textit{run\_a\_cycle()} function which
is called every cycle. Within the \textit{run\_a\_cycle()} function of each
component is the processing done by the component for every simulation cycle.
\textit{macsim\_c}, the simulation object, calls \textit{run\_a\_cycle()} for
the memory system, interconnect and cores in the simulation (see
    \textit{macsim.cc}). Each core in turn calls \textit{run\_a\_cycle()} for
the individual pipeline stages (see \textit{core.cc}).



\ignore{
Below is a list of key data structures used in MacSim along with a brief description:

\begin{description}

\item [core_c]

\item [frontend_c]
\item [bp_data_c]
\item [map_c]

\item [allocate_c]

\item [exec_c]
\item [dcu_c]

\item [retire_c]


\end{description}
}
