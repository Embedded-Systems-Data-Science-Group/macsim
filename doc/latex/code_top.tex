
\chapter{Overview of MacSim Code Structures}
\label{sec:codetop}

This section provides a brief overview of the organization of the code in
MacSim. Code in MacSim is organized into many classes, for a descpription of
the classes please refer to the doxygen documentation.

For each simulation, an instance of \textit{macsim\_c} is created.  In
\textit{macsim\_c}, there is an instance of \textit{process\_s} for each
application being simulated. An application can contain multiple threads each
represented by a \textit{thread\_s} structure. In case of GPU applications, an
instance of \textit{thread\_s} represents a warp. The threads of an application
can run on multiple cores each of which is an \textit{core\_c} object. For GPU
applications, each core represents an SM. Each core has a member variable for
each stage of the pipeline.  A \textit{core\_c} object also contains objects
for hardware structures such as ROB, I-cache, Shared Memory, branch predictor
and so on. Note that some of these components are valid for CPUs only, while
some others are valid for GPUs only.  According to the speficied memory
hierarchy, an instance of a class that derives from \textit{memory\_c} is
created. This object is a member of \textit{macsim\_c} and is accessed by all
cores in the simulation. The \textit{memory\_c} object includes objects for L1
D-cache, L2 and L3 caches and also DRAM controllers which are objects of
classes that derive from \textit{dram\_controller\_c}. 


\section{Simulator Execution}

All pipeline stages, units such as cache and memory controller, components such
as memory system and core implement the \textit{run\_a\_cycle()} function which
is called every cycle. Within the \textit{run\_a\_cycle()} function of each
component is the processing done by the component for every simulation cycle.
\textit{macsim\_c}, the simulation object, calls \textit{run\_a\_cycle()} for
the memory system, interconnect and cores in the simulation (see
    \textit{macsim.cc}). Each core in turn calls \textit{run\_a\_cycle()} for
the individual pipeline stages (see \textit{core.cc}).


\section{Adding a new DRAM Policy}


\section{Important simulator data structures}

\subsection{uop\_c}

An instance of \textit{uop\_c} defined in \textit{uop.h} is allocated (from a
    pool) for each uop and is populated with all the information about the uop.
Each uop is assigned a unique number, \textit{m\_unique\_num}, that is unique
across all uops in a core. Each uop is also assigned a unique number,
       \textit{m\_uop\_num}, that is unique across a uops of a thread. To
       identify a uop from a thread of a core, \textit{m\_core\_id},
       \textit{m\_thread\_id}, \textit{m\_uop\_num} member variables of the uop
       have to checked. Users can define new member variables to collect/track
       information as a uop moves through the pipeline. The \textit{uop\_c}
       object allocated for a uop is returned to the \textit{uop\_c} pool when
       the uop retires.  Whenever a new member variable is added to
       \textit{uop\_c} users should make sure that the member variable is
       initialized to a default value in \textit{uop\_c::init()}.

\subsection{core\_c}

Each core being simulated is represented by an object of \textit{core\_c}
declared in \textit{core.h}. Besides pointers to various pipeline stages and
hardware units, \textit{core\_c} tracks information such as how many threads
are assigned to the core, how many instructions have been fetched for each
thread and so on.


\subsection{mem\_req\_s}

For each memory request (L1 data cache miss, and write back requests), a
instance of \textit{mem\_req\_s} declared in \textit{memreq\_info.h} is
allocated. \textit{mem\_req\_s} contains all information relevant to a memory
request and an instance allocated to a request is freed only when the memory
request completes and data is filled in the cache. Users can define new member
variables to collect/track information as a memory request moves through the
memory hierarchy.


\subsection{drb\_entry\_s}

For each memory request that reaches the DRAM an instance of
\textit{drb\_entry\_s} declared in \textit{dram.h} is allocated. This instance
contains information about the bank, row and column of the DRAM request among
other details and is deallocated (freed) when the DRAM request is serviced.


\subsection{rob\_c/smc\_rob\_c}

\textit{rob\_c} declared in \textit{rob.h} and \textit{smc\_rob\_c} declared in
\textit{rob\_smc.h} are for ROBs in CPUs and GPUs respectively. uops are
allocated an entry in the ROB in the allocate stage and the allocated entry is
reclaimed when the uop retires.

\subsection{pool\_c}

\textit{pool\_c} defined \textit{utils.h} is a utility class meant for creating
memory pools. In Macsim it is used for creating pools of \textit{uop\_c},
       \textit{thread\_s} and so on.


\subsection{Latencies of instructions}

Latencies of X86 and PTX uops are defined in \textit{uoplatency\_x86.def} and
\textit{uoplatency\_ptx.def} files in \textit{trunk/def}. Users must edit these
files to modify the latencies of instructions. Note that for PTX instructions,
      the value defined in \textit{uoplatency\_ptx.def} is multiplied by the
      value of the knob \textit{PTX\_EXEC\_RATIO}.


\subsection{pqueue\_c}

\textit{pqueue\_c} is a utility class for varying the length of the simulated
pipeline. In MacSim, it is used to vary the length of the fetch (and decode)
  and allocation stages. At the time of creation of a \textit{pqueue\_c}
  object, user has to specify the latency of the pqueue in cycles.
  Objects/values enqueued into the are ready to be dequeued after speficied
  latency. Besides the object to be enqueued, the enqueue operation takes a
  priority value as argument as well. Different priority values can be used for
  objects enqueued in the same cycle to control which object gets dequeued
  first from the pqueue.


\subsection{Adding new dram policy}

To implement a new DRAM scheduling policy, a class that extends
\textit{dram\_controller\_c} has to be defined. This class should overload the
\textit{schedule()} function which should return the next DRAM request to be
serviced according to the policy. Please refer to class \textit{dc\_frfcfs\_s}
defined in dram.cc/h for a sample. In addition to the class, define a wrapper
function that returns an instance of the class (see \textit{dram.cc} for an
    example) and register the class with the DRAM factory (see \textit{see
      register\_functions() in \textit{macsim.cc}}. 


\subsection{Implementing a new instruction scheduler}

Define a class that extends \textit{schedule\_c} and overloads at least the
\textit{run\_a\_cycle()} function. Other functions may be overloaded to
maintain a consistent code structure across the different instruction
schedulers. After defining the new scheduler, add code to \textit{core.cc} to
instantiate your scheduler instead of one of the provided schedulers.

\subsection{Implementing a new policy for allocating thread blocks to GPU cores}

In \textit{process\_manager\_c::sim\_schedule\_thread\_block()} add (or modify)
code to determine the id of the block that will be assigned to the core based
on the new policy. The list of ready blocks can be obtained via
\textit{m\_block\_schedule\_info} member of the \textit{macsim\_c} object that
is created for the simulation.

\subsection{Implementing a new fetch policy}
\TODO{Jaekyu(?)}

\subsection{Implementing a new branch predictor}
\TODO{Jaekyu(?)}

\subsection{Implementing a few hardware prefetcher}
\TODO{Jaekyu}









\ignore{
Below is a list of key data structures used in MacSim along with a brief description:

\begin{description}

\item [core_c]

\item [frontend_c]
\item [bp_data_c]
\item [map_c]

\item [allocate_c]

\item [exec_c]
\item [dcu_c]

\item [retire_c]


\end{description}
}
