
\clearpage
\section{Setting up Parameters}
\label{sec:parameter}

This section provides sample examples for parameters to run different
types of simulations.



\subsection{\cpu Experiments}

\subsubsection{One \cpu Core}

Usually, we use large core configurations for specifying x86 cores
(however, it does not have to).

\smallskip
\begin{lstlisting}
// macsim-top/trunk/params/params_x86
num_sim_cores 1
num_sim_small_cores 0
num_sim_medium_cores 0
num_sim_large_cores 1
large_core_type x86
\end{lstlisting}
\smallskip

\subsubsection{Multiple \cpu cores}

Multiple core simulations are very similar to single core simulations,
but you need to specify the number of cores differently.

\smallskip
\begin{lstlisting}
// 4-core simulation
num_sim_cores 4
num_sim_small_cores 0
num_sim_medium_cores 0
num_sim_large_cores 4
large_core_type x86
repeat_trace 1
\end{lstlisting}
\smallskip


\subsubsection{2-way SMT x86 Core}

\SIM also supports the simultaneous multi-threading (SMT) features. 
There parameters are used for the SMT configurations: 
\textsf{max\_threads\_per\_core}, \textsf{max\_threads\_per\_medium\_core}, 
and \\\textsf{max\_thread\_per\_large\_core}. For example,

\smallskip
\begin{lstlisting}
// 1-core 2-way SMT configuration
num_sim_cores 1
num_sim_small_cores 0
num_sim_medium_cores 0
num_sim_large_cores 1
large_core_type x86
max_threads_per_large_core 2
repeat_trace 1
\end{lstlisting}
\smallskip



\subsection{\gpu Simulation}

Usually, we use small core configurations for \gpu cores. We provide
several pre-defined configuration files for NVIDIA architectures. Here
are the list of sample files.

\smallskip
\begin{lstlisting}
params_8800gt // NVIDIA GeForce 8800GT (G80 architecture)
params_gtx280 // NVIDIA GeForce GTX280 (GT200 architecture)
params_gtx465, params_gtx480 // NVIDIA GeForece GTX465, GTX480 (Fermi architecture)
\end{lstlisting}
\smallskip

\subsubsection{\gpu with One Application}
\smallskip
\begin{lstlisting}
// 12-SM simulations
num_sim_cores 12
num_sim_small_cores 12
core_type ptx
max_threads_per_core 80 // set the max number of warps 
\end{lstlisting}
\smallskip



\subsubsection{\gpu with Multiple Applications}

\smallskip
\begin{lstlisting}
// 12-SM simulations, 6 SMs for each application
num_sim_cores 12
num_sim_small_cores 12
core_type ptx
max_threads_per_core 80 // set the max number of warps 
max_num_core_per_appl 6 // 6 SMs for each application
repeat_trace 1 // for multi-program workload simulation
\end{lstlisting}
\smallskip


\subsection{Heterogeneous Simulation}

For the CPU-GPU heterogeneous simulations, we model the architecture
similar to Intel's Sandy Bridge~\cite{sandybridge}. However, we
model \gpu cores similar to NVIDIA Fermi~\cite{fermi} architecture. We
also provide sample files for the heterogeneous simulations.

\smallskip
\begin{lstlisting}
params_hetero_1_6 // 1-CPU, 6-GPU cores
params_hetero_4c_4g // 4-CPU, 4-GPU cores
\end{lstlisting}
\smallskip

\subsubsection{One CPU application + One GPU application}

Following example shows simple heterogeneous configuration.

\smallskip
\begin{lstlisting}
num_sim_cores 2
num_sim_small_cores 1
num_sim_medium_cores 0
num_sim_large_cores 1
core_type ptx
large_core_type x86
cpu_frequency 3
gpu_frequency 1.5
repeat_trace 1
\end{lstlisting}
\smallskip

Although above configurations set up the number of CPU and GPU cores
correctly, you need to setup each core types individually. Please
refer to sample files.

\subsubsection{Multiple CPU applications + Multiple GPU applications}

\smallskip
\begin{lstlisting}
num_sim_cores 8
num_sim_small_cores 4
num_sim_medium_cores 0
num_sim_large_cores 4
core_type ptx
large_core_type x86
cpu_frequency 3
gpu_frequency 1.5
repeat_trace 1
\end{lstlisting}
\smallskip


\subsection{Micro-architecture configuration}

\subsubsection{Number of blocks per core}

The Maximum number blocks per core will be determined by several
factors: 1) the number of threads per thread block, 2) the number of
register used, 3) the number of shared memory usage, and 4) NVIDIA
CUDA computing version. NVIDIA provides the occupancy calculator to
get the number. \SIM uses this value to allocate a number of thread
blocks in each core. Since this value is per-application
characteristic, it is not possible to use single value nor provide the
value individually. Instead, our approach is that using information
provided by GPUOcelot~\cite{ocelot} while generating traces. This
information is written in the trace file and \SIM uses it. However, we
also provide a way to override the value by
setting \textsf{max\_block\_per\_core\_super}.



\subsubsection{DRAM configuration}

We model three key components in the memory scheduler: timing
constraints, bandwidth, and scheduling policy. Following are key
parameters used in the \SIM.


\smallskip
\begin{lstlisting}
dram_frequency 0.8 // dram frequency
dram_bus_width 4 // dram bus width
dram_column 11 // column access (CL) latency
dram_activate 25 // row activate (RCD) latency
dram_precharge 10 // precharge (RP) latency
dram_num_mc 2 // number of memory controllers
dram_num_banks 8 // number of banks per controller
dram_num_channel 2 // number of dram channels per controller
dram_rowbuffer_size 2048 // row buffer size
dram_scheduling_policy FRFCFS // dram scheduling policy
\end{lstlisting}
\smallskip

Dram timings using three parameters: precharge ($t_{RP}$), activate
($t_{RCD}$), and column access ($t_{CL}$) latency. DRAM bandwidth is
modeled using three
parameters: \textsf{dram\_frequency}, \textsf{dram\_bus\_width},
and \textsf{dram\_num\_channel}. The maximum DRAM bandwidth can be
calculated using Equation~\ref{eq:bandwidth}.

\begin{equation}
\label{eq:bandwidth}
max\_bandwidth = dram\_frequency \times dram\_bus\_width \times dram\_num\_mc \times dram\_num\_channel 
\end{equation}

For example, the maximum bandwidth from above parameters can
be \textsf{800 MHz (0.8 GHz)} $\times$ \textsf{4 Bytes}
$\times$ \textsf{2 MCs} $\times$ \textsf{2 Channels}
= \textsf{12.8GB/s}. Finally, \SIM provides multiple DRAM scheduling
policies: FCFS (First-Come-First-Serve) and FR-FCFS (First-Ready
First-Come-First-Serve).


\subsubsection{Branch Predictor}


\ignore{
\subsubsection{Hardware Prefetching}

We provide stride hardware prefetcher~\cite{iac:spr04}. 
}

\subsubsection{Thread Fetch Policy}

% LocalWords:  macsim num sim SMT multi pre NVIDIA params GeForce gtx GTX ptx
% LocalWords:  GeForece Multipe SMs appl GPU cpu gpu CUDA GPUOcelot RCD mc GHz
% LocalWords:  precharge rowbuffer FRFCFS MCs FCFS Prefetching prefetcher
