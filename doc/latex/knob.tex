

\chapter{The Knobs}
\label{sec:knob}

To control various architectural parameters, we use multiple knob
variabls, which are defined in \textit{trunk/def/*.param.def}.

\section{How to add a new knob}
\label{sec:knob1}

A new knob variable can be defined in the following format:

\begin{Verbatim}
param<{variable used in the \SIM}, {variable used in the commandline}, {type}, {default value}>
\end{Verbatim}

We recommend first two arguments are same, but the former is in the
upper case and the latter is in the lower case. For example,

\begin{Verbatim}
param<L2_ASSOC, l2_assoc, int, 8>
\end{Verbatim}


However, one restriction is this new varialbe must be defined
in \textit{trunk/def} directory and the file name should
be \textit{*.param.def} to parse variables correctly.


\section{How to use a new knob in the simulator}

All knob variables will have \textit{KNOB\_} prefix. For example,
L2\_ASSOC in Section~\ref{sec:knob1} will be KNOB\_L2\_ASSOC in the
simulator code.

\subsection{Adding a New String Type Knob}
Several knobs especially setting policies use {\texttt string} type. 
Examples are branch predictor, instruction scheduler, dram scheduler, and llc setting. Please see {\texttt factory\_class.cc/h} and {\texttt bp.cc}. 





\section{How to apply different value to a knob variables}

There are two ways of modifying the default value of a knob variable.

\begin{enumerate}
  \item params.in - this file must be supplied to the macsim binary
  for the execution. You can find sample parameter files
  in \textit{trunk/params} directory. In this file, variable names and
  the values are paired in each line. For example,

\begin{Verbatim}
l1_assoc 8
l2_assoc 16
\end{Verbatim}


  \item command line - instead, you can also supply the knobs in the command line, for example,

\begin{Verbatim}
./macsim --l1_assoc=8 --l2_assoc=16
\end{Verbatim}

\end{enumerate}


