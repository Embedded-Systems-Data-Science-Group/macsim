

\chapter{Knobs and Parameters}
\label{sec:knob}

To control simulation and architectural parameters, we use knob variables which
are defined in \textit{trunk/def/*.param.def} files. The build process automatically 
converts the knob definitions in these files into c++ source that gets included in 
the compilation of the simulator binary.

\section{How to add a new knob}
\label{sec:knob1}

A new knob variable can be defined by adding a line in the following format in 
one of the\textit{param.def} files:

\begin{Verbatim}
param<{variable used in the \SIM}, {variable used in the commandline or params file}, {type}, {default value}>
\end{Verbatim}

We recommend that the first two arguments be the same, except that the former
be in upper case and the latter in lower case. For example,

\begin{Verbatim}
param<L2_ASSOC, l2_assoc, int, 8>
\end{Verbatim}


\ignore{ However, one restriction is this new variable must be defined in
\textit{trunk/def} directory and the file name should be \textit{*.param.def}
to parse variables correctly.  }

Knobs can be pretty much of any type, to support knobs of type other than the
basic data types you may have to add code to \textit{src/knob.h}. Knobs of type
string are already supported, this is helpful for specifying policies and
configurations such as branch predictor type, instruction scheduler type, dram
scheduling policy and so on. 


\section{How to use a new knob in the simulator}

In the simulator code, a knob variable is accessed by prefixing its name (in
uppercase) with \textit{KNOB\_}.  For example, L2\_ASSOC defined in the example
in Section~\ref{sec:knob1} can be accessed using the name KNOB\_L2\_ASSOC in
the simulator. To access the value of a knob use the \textit{getValue()}
function or just the name of the knob (this works because of operator
overloading).

\ignore { \subsection{Adding a New String Type Knob} Several knobs especially
setting policies use {\texttt string} type.  Examples are branch predictor,
instruction scheduler, dram scheduler, and llc setting. Please see {\texttt
factory\_class.cc/h} and {\texttt bp.cc}.  }



\section{How to assign values to knobs?}

When defining a knob, a default value for the knob has to be specified. If a
user does not set the value of a knob for a simulation, then the knob assumes
its default value. If a user wishes to change the value of a knob for a
simulation, there are two ways in which the user can accomplish this:


\ignore { \section{How to apply different value to a knob variables} There are
two ways of modifying the default value of a knob variable.  }

\begin{enumerate}

  \item edit params.in - this file is read by the macsim binary on startup for
parameter values.  Sample parameter files can be found in \textit{trunk/params}
directory. Each line in a parameter file consists of a knob name (in
lowercase) followed by the value of the parameter.  For example,

\begin{Verbatim}
l1_assoc 8
l2_assoc 16
\end{Verbatim}


  \item specify parameter values from the command line - an user can specify
  knobs and their values from the command line as shown below.

\begin{Verbatim}
./macsim --l1_assoc=8 --l2_assoc=16
\end{Verbatim}

\end{enumerate}

For a knob specified in both the params.in file and the command line, the value
specified in the command line takes priority over the value specified in the
params.in file.


%%%%%%%%%%%%%%%%%%%%%%%%%%%%%%%%%%%%%%%%%%%%%%%%%%%%%%%%%%%%%%%%%%%%%%%%
\section{Setting up Parameters}
\label{sec:parameter}
%%%%%%%%%%%%%%%%%%%%%%%%%%%%%%%%%%%%%%%%%%%%%%%%%%%%%%%%%%%%%%%%%%%%%%%%

This section provides sample examples for parameters to run different
types of simulations.


%%%%%%%%%%%%%%%%%%%%%%%%%%%%%%%%%%%%%%%%%%%%%%%%%%%%%%%%%%%%%%%%%%%%%%%%
\subsection{Repeating Trace}
%%%%%%%%%%%%%%%%%%%%%%%%%%%%%%%%%%%%%%%%%%%%%%%%%%%%%%%%%%%%%%%%%%%%%%%%

For multiple-application simulations, sometimes, early-terminating applications
have to be re-executed to model resource contention (cache, on-chip
    interconnection, and memory controller) until all applications finish. This
is a common methodology adopted in evaluating multi-program workloads and is
supported by \SIM also. To enable this feature, you have to turn on
repeate\_trace knob i.e. set repeat\_trace to 1. This can be done either via
params.in file or from the command line. 

\ignore
{

\begin{itemize}
  \item Specify the configuration in the command line by
  \begin{Verbatim}
  ./macsim --repeat_trace=1
  \end{Verbatim}

  \item Or you can write this in \textit{params.in} file
  \begin{Verbatim}
  repeat_trace 1
  \end{Verbatim}
\end{itemize}
}



%%%%%%%%%%%%%%%%%%%%%%%%%%%%%%%%%%%%%%%%%%%%%%%%%%%%%%%%%%%%%%%%%%%%%%%%
\subsection{\cpu Experiments}
%%%%%%%%%%%%%%%%%%%%%%%%%%%%%%%%%%%%%%%%%%%%%%%%%%%%%%%%%%%%%%%%%%%%%%%%


%%%%%%%%%%%%%%%%%%%%%%%%%%%%%%%%%%%%%%%%%%%%%%%%%%%%%%%%%%%%%%%%%%%%%%%%
\subsubsection{One \cpu Core}
%%%%%%%%%%%%%%%%%%%%%%%%%%%%%%%%%%%%%%%%%%%%%%%%%%%%%%%%%%%%%%%%%%%%%%%%

Usually, we configures cores of type large as x86 cores, however, this is not
mandatory.

\begin{Verbatim}
// macsim-top/trunk/params/params_x86
num_sim_cores 1
num_sim_small_cores 0
num_sim_medium_cores 0
num_sim_large_cores 1
large_core_type x86
\end{Verbatim}


%%%%%%%%%%%%%%%%%%%%%%%%%%%%%%%%%%%%%%%%%%%%%%%%%%%%%%%%%%%%%%%%%%%%%%%%
\subsubsection{Multiple \cpu cores}
%%%%%%%%%%%%%%%%%%%%%%%%%%%%%%%%%%%%%%%%%%%%%%%%%%%%%%%%%%%%%%%%%%%%%%%%

Multiple core simulations are very similar to single core simulations, but you
have to specify the number of cores as greater than 1.


\begin{Verbatim}
// 4-core simulation
num_sim_cores 4
num_sim_small_cores 0
num_sim_medium_cores 0
num_sim_large_cores 4
large_core_type x86
repeat_trace 1
\end{Verbatim}


%%%%%%%%%%%%%%%%%%%%%%%%%%%%%%%%%%%%%%%%%%%%%%%%%%%%%%%%%%%%%%%%%%%%%%%%

\subsubsection{2-way SMT x86 Core}

\SIM also supports the simultaneous multi-threading (SMT) features.
These parameters are used for the SMT configurations:

\Verb+ max_threads_per_core+, \Verb+max_threads_per_medium_core+, and
\Verb+max_thread_per_large_core+. 

\noindent
For example,

\begin{Verbatim}
// 1-core 2-way SMT configuration
num_sim_cores 1
num_sim_small_cores 0
num_sim_medium_cores 0
num_sim_large_cores 1
large_core_type x86
max_threads_per_large_core 2
repeat_trace 1
\end{Verbatim}



%%%%%%%%%%%%%%%%%%%%%%%%%%%%%%%%%%%%%%%%%%%%%%%%%%%%%%%%%%%%%%%%%%%%%%%%
\subsection{\gpu Simulation}
%%%%%%%%%%%%%%%%%%%%%%%%%%%%%%%%%%%%%%%%%%%%%%%%%%%%%%%%%%%%%%%%%%%%%%%%

Usually, we configure cores of type small as \gpu cores. We provide
several pre-defined configuration files for NVIDIA architectures. Here
is the list of sample files.

\begin{Verbatim}
params_8800gt // NVIDIA GeForce 8800GT (G80 architecture)
params_gtx280 // NVIDIA GeForce GTX280 (GT200 architecture)
params_gtx465, params_gtx480 // NVIDIA GeForece GTX465, GTX480 (Fermi architecture)
\end{Verbatim}


%%%%%%%%%%%%%%%%%%%%%%%%%%%%%%%%%%%%%%%%%%%%%%%%%%%%%%%%%%%%%%%%%%%%%%%%
\subsubsection{\gpu with One Application}
%%%%%%%%%%%%%%%%%%%%%%%%%%%%%%%%%%%%%%%%%%%%%%%%%%%%%%%%%%%%%%%%%%%%%%%%

\begin{Verbatim}
// 12-SM simulations
num_sim_cores 12
num_sim_small_cores 12
core_type ptx
max_threads_per_core 80 // set the max number of warps per SM
\end{Verbatim}


%%%%%%%%%%%%%%%%%%%%%%%%%%%%%%%%%%%%%%%%%%%%%%%%%%%%%%%%%%%%%%%%%%%%%%%%
\subsubsection{\gpu with Multiple Applications}
%%%%%%%%%%%%%%%%%%%%%%%%%%%%%%%%%%%%%%%%%%%%%%%%%%%%%%%%%%%%%%%%%%%%%%%%

\begin{Verbatim}
// 12-SM simulations, 6 SMs for each application
num_sim_cores 12
num_sim_small_cores 12
core_type ptx
max_threads_per_core 80 // set the max number of warps 
max_num_core_per_appl 6 // 6 SMs for each application
repeat_trace 1 // for multi-program workload simulation
\end{Verbatim}


%%%%%%%%%%%%%%%%%%%%%%%%%%%%%%%%%%%%%%%%%%%%%%%%%%%%%%%%%%%%%%%%%%%%%%%%
\subsection{Heterogeneous Simulation}
%%%%%%%%%%%%%%%%%%%%%%%%%%%%%%%%%%%%%%%%%%%%%%%%%%%%%%%%%%%%%%%%%%%%%%%%

For the CPU-GPU heterogeneous simulations, we model the architecture
similar to Intel's Sandy Bridge~\cite{sandybridge}. However, we
model \gpu cores similar to NVIDIA Fermi~\cite{fermi} architecture. We
also provide sample files for the heterogeneous simulations.

\begin{Verbatim}
params_hetero_1_6 // 1-CPU, 6-GPU cores
params_hetero_4c_4g // 4-CPU, 4-GPU cores
\end{Verbatim}


%%%%%%%%%%%%%%%%%%%%%%%%%%%%%%%%%%%%%%%%%%%%%%%%%%%%%%%%%%%%%%%%%%%%%%%%
\subsubsection{One CPU application + One GPU application}
%%%%%%%%%%%%%%%%%%%%%%%%%%%%%%%%%%%%%%%%%%%%%%%%%%%%%%%%%%%%%%%%%%%%%%%%

Following example shows a simple heterogeneous configuration.

\begin{Verbatim}
num_sim_cores 2
num_sim_small_cores 1
num_sim_medium_cores 0
num_sim_large_cores 1
core_type ptx
large_core_type x86
cpu_frequency 3
gpu_frequency 1.5
repeat_trace 1
\end{Verbatim}


Although above configurations set up the number of CPU and GPU cores
correctly, you need to setup each core types individually. Please
refer to sample files.

%%%%%%%%%%%%%%%%%%%%%%%%%%%%%%%%%%%%%%%%%%%%%%%%%%%%%%%%%%%%%%%%%%%%%%%%
\subsubsection{Multiple CPU applications + Multiple GPU applications}
%%%%%%%%%%%%%%%%%%%%%%%%%%%%%%%%%%%%%%%%%%%%%%%%%%%%%%%%%%%%%%%%%%%%%%%%

\begin{Verbatim}
num_sim_cores 8
num_sim_small_cores 4
num_sim_medium_cores 0
num_sim_large_cores 4
core_type ptx
large_core_type x86
cpu_frequency 3
gpu_frequency 1.5
repeat_trace 1
\end{Verbatim}


%%%%%%%%%%%%%%%%%%%%%%%%%%%%%%%%%%%%%%%%%%%%%%%%%%%%%%%%%%%%%%%%%%%%%%%%
\subsection{Micro-architecture configuration}
%%%%%%%%%%%%%%%%%%%%%%%%%%%%%%%%%%%%%%%%%%%%%%%%%%%%%%%%%%%%%%%%%%%%%%%%

%%%%%%%%%%%%%%%%%%%%%%%%%%%%%%%%%%%%%%%%%%%%%%%%%%%%%%%%%%%%%%%%%%%%%%%%
\subsubsection{Number of Thread Blocks Per Core}
%%%%%%%%%%%%%%%%%%%%%%%%%%%%%%%%%%%%%%%%%%%%%%%%%%%%%%%%%%%%%%%%%%%%%%%%

The Maximum number blocks per core will be determined by several
factors: 1) the number of threads per thread block, 2) the number of
register used, 3) the number of shared memory usage, and 4) NVIDIA
CUDA computing version. NVIDIA provides the occupancy calculator to
get the number. \SIM uses this value to allocate a number of thread
blocks in each core. Since this value is per-application
characteristic, it is not possible to use single value nor provide the
value individually. Instead, our approach is that using information
provided by GPUOcelot~\cite{ocelot} while generating traces. This
information is written in the trace file and \SIM uses it. However, we
also provide a way to override the value by
setting \Verb+max_block_per_core_super+.



%%%%%%%%%%%%%%%%%%%%%%%%%%%%%%%%%%%%%%%%%%%%%%%%%%%%%%%%%%%%%%%%%%%%%%%%
\subsubsection{Cache Configuration}
%%%%%%%%%%%%%%%%%%%%%%%%%%%%%%%%%%%%%%%%%%%%%%%%%%%%%%%%%%%%%%%%%%%%%%%%

The cache can be configured by following parameters.

\begin{Verbatim}
l{1,2}_{small, medium, large}_num_set // number of sets
l{1,2}_{small, medium, large}_assoc // associativity
l{1,2}_{small, medium, large}_line_size // cache line size
l{1,2}_{small, medium, large}_num_bank // number of banks  
l{1,2}_{small, medium, large}_latency // cache latency
l{1,2}_{small, medium, large}_bypass // cache bypass (if set, always miss)
num_l3 // number of l3 cache tiles
l3_num_Set // number of l3 cache sets
l3_assoc // l3 associativity
l3_line_size // l3 line size
l3_num_bank // l3 number of banks
l3_latency // l3 latency
l{1,2,3}_{read,write}_port // the number of read / write port
icache_num_set  8 // 4KB I-cache 
icache_assoc   8 // I cache set associativity 
\end{Verbatim}


The effective cache size can be calculated by the
Equation~\ref{eq:cachesize}.

\begin{equation}
\label{eq:cachesize}
cache\_size = num\_set \times assoc \times line\_size \times num\_tiles (l3 only, otherwise 1)
\end{equation}


\noindent
For example, to configure a 16-way 1MB cache with 64B line size,

\Verb+ 1MB cache = 256 sets * 16 way *+ \Verb+64B * 4 tiles+
%\Verb+\textsf{16 way} $\times$ \textsf{64 B}+ $\times$
%\Verb+\textsf{4 tiles}+. 

\noindent
Also, the cache latency is determined by several factors including the
size, technology, and the number of ports. Cacti~\cite{cacti} can be
used to model accurate cache latency. The cache line size is set as
64B by default. Although the cache line size can be any power of 2,
all cache levels should have the same size.


%%%%%%%%%%%%%%%%%%%%%%%%%%%%%%%%%%%%%%%%%%%%%%%%%%%%%%%%%%%%%%%%%%%%%%%%
\subsubsection{DRAM configuration}
\label{sec:param-dram}
%%%%%%%%%%%%%%%%%%%%%%%%%%%%%%%%%%%%%%%%%%%%%%%%%%%%%%%%%%%%%%%%%%%%%%%%

We model three key components in the memory scheduler: timing
constraints, bandwidth, and scheduling policy. Following are key
parameters used in the \SIM.


\begin{Verbatim}
dram_frequency 0.8 // dram frequency
dram_bus_width 4 // dram bus width
dram_column 11 // column access (CL) latency
dram_activate 25 // row activate (RCD) latency
dram_precharge 10 // precharge (RP) latency
dram_num_mc 2 // number of memory controllers
dram_num_banks 8 // number of banks per controller
dram_num_channel 2 // number of dram channels per controller
dram_rowbuffer_size 2048 // row buffer size
dram_scheduling_policy FRFCFS // dram scheduling policy
\end{Verbatim}


Dram timings using three parameters: precharge ($t_{RP}$), activate
($t_{RCD}$), and column access ($t_{CL}$) latency. DRAM bandwidth is
modeled using three
parameters: \Verb+dram_frequency+, \Verb+dram_bus_width+,
and \Verb+dram_num_channel+. The maximum DRAM bandwidth can be
calculated using Equation~\ref{eq:bandwidth}.

\begin{equation}
\label{eq:bandwidth}
max\_bandwidth = dram\_frequency \times dram\_bus\_width \times dram\_num\_mc \times dram\_num\_channel 
\end{equation}

For example, the maximum bandwidth from above parameters can be
\Verb+ 800 MHz (0.8 GHz)} * 4 Bytes+
  \Verb+* 2 MCs * 2 Channels = 12.8GB/s+ . Finally, \SIM provides
  multiple DRAM scheduling policies: FCFS (First-Come-First-Serve) and
  FR-FCFS (First-Ready First-Come-First-Serve).


%%%%%%%%%%%%%%%%%%%%%%%%%%%%%%%%%%%%%%%%%%%%%%%%%%%%%%%%%%%%%%%%%%%%%%%%
\subsubsection{Setting \cpu Core Parameters}
%%%%%%%%%%%%%%%%%%%%%%%%%%%%%%%%%%%%%%%%%%%%%%%%%%%%%%%%%%%%%%%%%%%%%%%%

By default these configurations are applied to small cores. 

\begin{Verbatim}
width   2 // pipeline width (the entire pipelines use the same width) 
fetch_latency 5 // front-end depth 
alloc_latency 5 // decode/allocation depth  
bp_dir_mech   gshare 
bp_hist_length 14 // branch history length 
isched_rate 4  // # of integer instructions that can be executed per cycle 
msched_rate 2  // # of memory instructions that can be executed per cycle 
fsched_rate  2  // # of FP instructions that can be executed per cycle 
schedule  io, ooo // instruction scheduling policy 
rob_size    96  // ROB size
fetch_policy rr // SMT(MT) thread fetch policy  by default: round-robin 
\end{Verbatim}


%%%%%%%%%%%%%%%%%%%%%%%%%%%%%%%%%%%%%%%%%%%%%%%%%%%%%%%%%%%%%%%%%%%%%%%%
\subsubsection{Setting GPU Core Parameters}
%%%%%%%%%%%%%%%%%%%%%%%%%%%%%%%%%%%%%%%%%%%%%%%%%%%%%%%%%%%%%%%%%%%%%%%%
{\textbf GPU Core Parameters} 

\TODO{Question: is constant cache size is per SM? How does a core share the constant cache, texture cache, what is ptx\_exec\_ratio? have vs need} 


\begin{Verbatim}
schedule_ratio 4 \\ Every 4th cycle, one thread (warp) is scheduled 
fetch_ratio  4 \\ Every 4th cycle, one thread (warp) is fetched 
gpu_sched   1 \\ set GPU scheduler 
ptx_common_cache \\ 
const_cache_size  1024 \\ 1024B constant cache  
texture_cache_size 1024  \\ 1024B texture cache
shared_mem_size 4096  \\ 4096B shared memory size 
ptx_exec_ratio  \\ 
num_warp_scheduler \\ the number of warps (threads) can be scheduled  
\end{Verbatim}








\ignore{
\subsubsection{Hardware Prefetching}

We provide stride hardware prefetcher~\cite{iac:spr04}. 
}



% LocalWords:  macsim num sim SMT multi pre NVIDIA params GeForce gtx GTX ptx
% LocalWords:  GeForece Multipe SMs appl GPU cpu gpu CUDA GPUOcelot RCD mc GHz
% LocalWords:  precharge rowbuffer FRFCFS MCs FCFS Prefetching prefetcher alloc
% LocalWords:  icache Microarchitecture bp dir mech gshare isched msched fsched
% LocalWords:  FP io ooo rr th sched const mem





