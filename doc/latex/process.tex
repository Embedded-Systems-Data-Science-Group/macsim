
\clearpage
\section{Process Manager/Thread Scheduler}

MacSim uses a common Process Manager/Thread Scheduler for both CPU threads and
GPU warps. For each application that is to be simulated, the Process Manager
creates a process and also creates the threads/warps in the application. Based
on the simulation configuration, the Process Manager assigns cores to each
application, these cores are dedicated to the application. In case of CPU
applications, only the main thread of the application is launched first. The
trace config that is input to the simulator specifies in terms of instructions
executed by the main thread when each child thread should be started. When a
child thread becomes ready for execution, the Process Manager is responsible
for assigning the child threads to cores. In case of GPU applications, the
Process Manager creates warps and forms thread blocks from these warps. The
thread blocks are assigned to cores according to the maximum number of blocks
supported by the core. Though the term core is commonly used for both X86 cores
and GPU cores, we want to clarify that a X86 thread can run only on a core
specified as X86 and a warp/block can run only a core specified as a GPU core.
Threads/warps once assigned to a core, remain attached to the core until they
terminate. When a thread or a warp terminates, the Process Manager is invoked
again for updating bookkeeping information. When it is determined that an
application can terminated, based on the simulation parameters, the Process
Manager could repeat the simulation of the application until the termination
condition is met.



Some of the key data structures used by the Process Manager are:

\begin{description}

\item [process\_s] For each application that is to be simulated, the process manager
creates a instance of type process\_s. This structure includes information such
as process id, start information of threads, number of threads (warps) in
application, number of threads (warps) created, number of threads (warps)
  terminated, list of cores on which the application can run and so on.

\item [thread\_s] This structure is analogous to the task struct maintained by an
operating system kernel. Each CPU thread and GPU warp has an instance of
thread\_s structure. This structure includes fields for thread id, block id,
  pointer to trace file, process to which the thread belongs and other fields. 

\item [block\_schedule\_info\_s] Bookkeeping structure representing thread blocks in
a GPGPU application. This structure contains fields for number of warps in a
block, number of terminated warps, core to which the block is assigned and so
on.

\item [thread\_trace\_info\_node\_s] Wrapper structure around thread\_s used by
Process manager to track threads/warps yet to be assigned cores.

\end{description}

