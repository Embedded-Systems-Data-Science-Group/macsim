%%%%%%%%%%%%%%%%%%%%%%%%%%%%%%%%%%%%%%%%%%%%%%%%%%%%%%%%%%%%%%%%%%%%%%%%
\chapter{Adding to \SIM}
\label{sec:add_macsim}
%%%%%%%%%%%%%%%%%%%%%%%%%%%%%%%%%%%%%%%%%%%%%%%%%%%%%%%%%%%%%%%%%%%%%%%%
In this chapter procedure to add new policies, predictors and prefetcher 
to \SIM are explained.

%%%%%%%%%%%%%%%%%%%%%%%%%%%%%%%%%%%%%%%%%%%%%%%%%%%%%%%%%%%%%%%%%%%%%%%%
\section{New DRAM Policy}
%%%%%%%%%%%%%%%%%%%%%%%%%%%%%%%%%%%%%%%%%%%%%%%%%%%%%%%%%%%%%%%%%%%%%%%%


To implement a new DRAM scheduling policy, a class that extends
\textit{dram\_controller\_c} has to be defined. This class should
overload the \textit{schedule()} function which should return the next
DRAM request to be serviced according to the policy. Please refer to
class \textit{dc\_frfcfs\_s} defined in dram.cc/h for a sample. In
addition to the class, define a wrapper function that returns an
instance of the class (see \textit{dram.cc} for an example) and
register the class with the DRAM factory (see \textit{see
register\_functions() in \textit{macsim.cc}}.


%%%%%%%%%%%%%%%%%%%%%%%%%%%%%%%%%%%%%%%%%%%%%%%%%%%%%%%%%%%%%%%%%%%%%%%%
\section{New Instruction Scheduler}
%%%%%%%%%%%%%%%%%%%%%%%%%%%%%%%%%%%%%%%%%%%%%%%%%%%%%%%%%%%%%%%%%%%%%%%%

Define a class that extends \textit{schedule\_c} and overloads at least the
\textit{run\_a\_cycle()} function. Other functions may be overloaded to
maintain a consistent code structure across the different instruction
schedulers. After defining the new scheduler, add code to \textit{core.cc} to
instantiate the new scheduler instead of one of the provided schedulers.

%%%%%%%%%%%%%%%%%%%%%%%%%%%%%%%%%%%%%%%%%%%%%%%%%%%%%%%%%%%%%%%%%%%%%%%%
\section{New policy for assigning thread blocks to GPU cores}
%%%%%%%%%%%%%%%%%%%%%%%%%%%%%%%%%%%%%%%%%%%%%%%%%%%%%%%%%%%%%%%%%%%%%%%%

In \textit{process\_manager\_c::sim\_schedule\_thread\_block()} add (or modify)
code to determine the id of the block that will be assigned to the core based
on the new policy. The list of ready blocks can be obtained via
\textit{m\_block\_schedule\_info} member of the \textit{macsim\_c} object that
is created for the simulation.

%%%%%%%%%%%%%%%%%%%%%%%%%%%%%%%%%%%%%%%%%%%%%%%%%%%%%%%%%%%%%%%%%%%%%%%%
\section{New Fetch Policy}
\label{sec:modify:fetch}
%%%%%%%%%%%%%%%%%%%%%%%%%%%%%%%%%%%%%%%%%%%%%%%%%%%%%%%%%%%%%%%%%%%%%%%

There are several steps to add a new thread fetching policy. We will
provide an example with round-robin fetch policy, which is already
provided by \SIM.

\begin{enumerate}[Step 1.]
\item Define a new fetch policy in \Verb+class frontend_c+.
\begin{Verbatim}
// frontend.h
class frontend_c {
  ...
  int fetch_rr(void);
  ...
}
\end{Verbatim}

\item Implement this new policy.

\item Register this new policy in \Verb+macsim.cc+.
\begin{Verbatim}
fetch_factory_c::get()->register_class("rr", fetch_factory);
\end{Verbatim}


\item Assign this policy to \Verb+MT_fetch_Scheduler+ in \Verb+frontend.cc+.
\begin{Verbatim}
if (policy == ``rr'') {
  MT_fetch_scheduler = &frontend_c::fetch_rr;
}
\end{Verbatim}

\item Enable this new branch predictor by setting \Verb+KNOB_FETCH_POLICY+.
\begin{Verbatim}
./macsim --fetch_policy=rr     -- in the commandline
fetch_policy rr                -- in the params.in
\end{Verbatim}
\end{enumerate}


%%%%%%%%%%%%%%%%%%%%%%%%%%%%%%%%%%%%%%%%%%%%%%%%%%%%%%%%%%%%%%%%%%%%%%%%
\section{New Branch Predictor}
%%%%%%%%%%%%%%%%%%%%%%%%%%%%%%%%%%%%%%%%%%%%%%%%%%%%%%%%%%%%%%%%%%%%%%%%

There are several steps to add a new branch predictor. We will provide
an example with gshare~\cite{mcf93} branch predictor, which is already
provided by \SIM.

\begin{enumerate}[Step 1.]
  \item Define a new class which inherits \Verb+class bp_dir_base_c+
\begin{Verbatim}
class bp_gshare_c : public bp_dir_base_c {};
\end{Verbatim}

  \item Define and implement other necessary features of a new branch
    predictor. Especially, override \Verb+pred()+, \Verb+update()+,
    and \Verb+recover()+ functions.
\begin{Verbatim}
uns8 bp_gshare_c::pred(uop_c* uop) {...}
void bp_gshare_c::update(uop_c* uop) {...}

\end{Verbatim}

  \item Add a policy to the branch predictor factory in
    \Verb+macsim.cc+ and to the allocator in \Verb+bp.cc+.

\begin{Verbatim}
// macsim.cc macsim_c::register_functions()
bp_factory_c::get()->register_class(``gshare'', default_bp);

// bp.cc
bp_dir_base_c* default_bp(macsim_c* simBase)
{
  ...
  else if (bp_type == ''gshare'')
    new_bp = new bp_gshare_c(simBase);
  ...
}
\end{Verbatim}

  \item Enable this new branch predictor by setting \Verb+KNOB_BP_DIR_MECH+.
\begin{Verbatim}
./macsim --bp_dir_mech=gshare     -- in the commandline
bp_dir_mech gshare                -- in the params.in
\end{Verbatim}
\end{enumerate}


%%%%%%%%%%%%%%%%%%%%%%%%%%%%%%%%%%%%%%%%%%%%%%%%%%%%%%%%%%%%%%%%%%%%%%%%
\section{New Hardware Prefetcher}
%%%%%%%%%%%%%%%%%%%%%%%%%%%%%%%%%%%%%%%%%%%%%%%%%%%%%%%%%%%%%%%%%%%%%%%%

There are several steps to add a new branch predictor. We will provide
an example with a stride~\cite{iac:spr04} branch predictor, which is already
provided by \SIM.

\begin{enumerate}[Step 1.]
  \item Define a new class which inherits \Verb+class pref_base_c+
\begin{Verbatim}
class pref_stride_c : public pref_base_c {};
\end{Verbatim}

  \item Define and implement other necessary features of a new
    hardware prefetcher. Especially, override several function.
\begin{Verbatim}
void init_func(int);                            // on initiation of prefetch request
void done_func();                               // when a prefetch request is installed
void l1_miss_func(int, Addr, Addr, uop_c*);     // on L1 miss
void l1_hit_func(int, Addr, Addr, uop_c*);      // on L1 hit
void l1_pref_hit_func(int, Addr, Addr, uop_c*); // on L1 prefetch hit
void l2_miss_func(int, Addr, Addr, uop_c*);     // on L2 miss
void l2_hit_func(int, Addr, Addr, uop_c*);      // on L2 hit
void l2_pref_hit_func(nt, Addr, Addr, uop_c*);  // on L2 prefetch hit
\end{Verbatim}

  \item Add a prefetcher to the prefetcher factory in \Verb+pref_factory.cc+.

\begin{Verbatim}
// pref_factory.cc
void pref_factory(...)
{
  pref_base_c* pref_stride = new pref_stride_c(...);
  pref_table.push_back(pref_stride);
  ...
}
\end{Verbatim}

  \item Enable this new hardware prefetcher by setting
    \Verb+KNOB_PREF_STRIDE_ON+ (please refer to pref\_stride.cc
    pref\_stride\_c::pref\_stride\_c()). \textit{Note that the way we
      simulate hardware prefetchers are not same as other modules. We
      will provide the detail of hardware prefetcher in
      Section~\ref{}}.
\begin{Verbatim}
switch (type) {
  case UNIT_SMALL:
    knob_enable = *m_simBase->m_knobs->KNOB_PREF_STRIDE_ON;
      break;
    case UNIT_MEDIUM:
      knob_enable = *m_simBase->m_knobs->KNOB_PREF_STRIDE_ON_MEDIUM_CORE;
      break;
    case UNIT_LARGE:
      knob_enable = *m_simBase->m_knobs->KNOB_PREF_STRIDE_ON_LARGE_CORE;
      break;
}

./macsim --pref_stride_on[,medium_core, large_core]=1
pref_stride_on[,medium_core, large_core] 1
\end{Verbatim}
\end{enumerate}
