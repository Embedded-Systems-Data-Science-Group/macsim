

\clearpage
\appendix

\section{Coding Style Guideline}


\subsection{Naming Conventions}

Names should be as descriptive as possible, do not hesitate to use
long names.  Shorten names only when the purpose of the name can still
be understood clearly.


\subsubsection{Classes}

\begin{itemize}

  \item \textbf{Class names} should use lower case alphabets only.  If the name
consists of multiple words, then the words should be separated using
the underscore character. Class names should end with c.  Eg.

\smallskip
\begin{lstlisting}
class  my_class_c;
\end{lstlisting}
\smallskip


  \item \textbf{Class member variables} should use the prefix m .  Member variable
names should use lower case alphabets only, multiple words should be
separated using the underscore character.  Eg.

\smallskip
\begin{lstlisting}
int  m_instance_variable;
\end{lstlisting}
\smallskip


  \item \textbf{Accessor i.e, get/set, member functions} should use
the names of the corresponding member variables without the m
suffix. In addition, the set functions should use the prefix set ,
while no prefix is needed for the get functions.  Eg.


\smallskip
\begin{lstlisting}
int  m_my_variable;
...
...
void  set_my_variable(int);
int  my_variable();
\end{lstlisting}
\smallskip


  \item \textbf{Other member functions} should always start with a
verb, multiple words should be separated with the underscore character
and no upper case letters are allowed.  Eg.

\smallskip
\begin{lstlisting}
int  access_cache();
\end{lstlisting}
\smallskip


  \item Example class declaration

\smallskip
\begin{lstlisting}
class  my_class_c  {
  private:
    int  m_var_one;
    bool  m_var_two;
    ...
    ...

    void  set_var_one(int);
    int  var_one();
    ...
    int  check_values();
};
\end{lstlisting}
\smallskip

\end{itemize}





\subsubsection{Structures}

\begin{itemize}

  \item Structure names should use lower case alphabets only. If the
name consists of multiple words, then the words should be separated
using the underscore character. Structure names should end with s. Eg.

\smallskip
\begin{lstlisting}
struct  my_struct_s;
\end{lstlisting}
\smallskip

  \item Do not use any prefix or suffix for structure member
  variables.

  \item Example structure declaration

\smallskip
\begin{lstlisting}
typedef struct my_type_s  {
  type field_one;
  type field_two;
}  my_type_s;
\end{lstlisting}
\smallskip

\end{itemize}





\subsubsection{Other Types}

Other user defined types should use only lower case letters with
multiple words being separated by the underscore character. These
types should use the suffix t.  Eg.

\smallskip
\begin{lstlisting}
typedef  std::map<int,  std::list<int>  > my_type_t;
\end{lstlisting}
\smallskip


\subsubsection{Global Variables}

Global variable names should use the prefix g and use lower case
letters only, multiple words should be separated by the underscore
character.  Eg.

\smallskip
\begin{lstlisting}
int  g_my_global_variable;
\end{lstlisting}
\smallskip


\subsubsection{Local Variables}

Local variables follow the same convention as global variables, except
that they do not require the g prefix.  Eg.

\smallskip
\begin{lstlisting}
int  my_local_variable;
\end{lstlisting}
\smallskip


\subsubsection{Constant  Variables}

Constant variables follow the same convention as global variables,
except that they use the prefix k , instead of g .  Eg.

\smallskip
\begin{lstlisting}
const  int  k_my_const  = 25;
\end{lstlisting}
\smallskip


\subsubsection{Macros}

Macro names should use upper case letters only with multiple words
being separated my underscores. Eg.

\smallskip
\begin{lstlisting}
#define  MAX_NUM_THREADS  8192
\end{lstlisting}
\smallskip




\subsection{Formatting and Readability}

\subsubsection{Indentation}

Use only spaces for indentation, do not use tabs.  The indentation
must be two spaces wide.


\subsubsection{Variable Declaration}

Variables should be declared at the point in the code where they are
needed, and should be assigned an initial value at the time of
declaration.


\subsubsection{Control Statements and Function Definitions}

\begin{itemize}

  \item For \textit{if}, \textit{for}, \textit{while}
and \textit{switch} statements, there should be a single single space
between the keyword and the opening parenthesis.  The opening curly
brace should appear on the same line as the statement and there should
be a single space between the closing parenthesis and the opening
curly brace.  The opening curly brace following a do statement must
also follow the same rules.  The \textit{while} of a \textit{do while}
loop should be on the same line as the closing parenthesis of the loop
body and should be separated from the closing parenthesis by a single
space.  Eg.

\smallskip
\begin{lstlisting}
//correct
if (condition)  { 
  statement 1; 
  statement 2;
}

//correct
for  (int  ii = 0;  ii < N;  ++ii)  {
  statement 1;
  statement 2;
}

//correct
while  (condition)  { 
  statement 1; 
  statement 2;
}

//correct
switch  (variable)  {
  condition 1: 
    statement 1; 
    break;
  condition 2: 
    statement 2; 
    break;
  default: 
    statement 3; 
    break;
}

//correct 
do {
  statement 1;
  statement 2;
} while (condition);
\end{lstlisting}
\smallskip



  \item \textit{else} statements should be on the line following the
closing parenthesis of the corresponding if block. There should be a
single space between the else keyword and the opening curly brace of
the else block. The same rules apply for \textit{else if} statements
as well.  Eg.

\smallskip
\begin{lstlisting}
//correct
if (test_cond)  {
...
}
else  {
...

}

//correct
if (test_cond)  {
...
}
else  if (another_cond)  {
...
}
\end{lstlisting}
\smallskip

  \item When defining functions, there should be no spaces between the
function name and the opening parenthesis, there should be a single
space between the closing parenthesis and the opening curly brace which
should appear on the same line as the function name.  Eg.

\smallskip
\begin{lstlisting}
//correct
int  my_func_defn(int  arg_a,  int  arg_b)  {
  ...
}
\end{lstlisting}
\smallskip
\end{itemize}


\subsubsection{Horizontal Spacing}

Except for when using the operators listed below use a single space
between the operator and the operands (note the slight difference in
using the comma operator).  For the operands listed below, spaces
should not be used.  List of operators which do not need horizontal
spacing:

\smallskip
\begin{lstlisting}
() [] ->    ++ -- + - ! ~  (type)  *  &   sizeof
\end{lstlisting}
\smallskip

Eg.

\smallskip
\begin{lstlisting}
i = j + k * i;     // correct 
i = j+k*i;         // wrong

array[i]  = j;     // correct 
array [i] = j;     // wrong

my_func(a, b, c);  // correct 
my_func(a,b,c);    // wrong
\end{lstlisting}
\smallskip



\subsubsection{Statements}

Each statement should be on its own line, this applies to variable
declarations as well. Eg.

\smallskip
\begin{lstlisting}
int i = 0;          // correct 
int j = 0;          // correct
int i = 0,  j = 0;  // wrong 
i++;                // correct
j++;                // correct

i++;  j++;          // wrong 
i = 0;  j = 0;      // Wrong
\end{lstlisting}
\smallskip



\subsubsection{Column Length}

Lines should not exceed more than 80 characters in length.  If a line
is longer than 80 characters, split it into multiple lines.


\subsection{Header Files and Include Directives}

\begin{itemize}

  \item All header files should have an ``include guard'' to prevent
  accidental inclusion of the file multiple times in a single
  compilation unit.  The include guard should be named after the file
  name.  Eg.

\smallskip
\begin{lstlisting}
// in  pref_common.h
#ifndef  PREF_COMMON_H  #define  PREF_COMMON_H
...
...
#endif
\end{lstlisting}
\smallskip


  \item Avoid including header files in other header files by using
  forward declaration.

  \item All include directives should be grouped with system files
  listed first, followed by normal simulator header files.
\end{itemize}

\subsection{Comments}

Use Doxygen style comments. More to be added.


\subsection{Other Rules}

\begin{itemize}

  \item Avoid C style coding, try to build and use classes as much as
    possible.

  \item Define types which will not have functions other that accessor
    functions (and constructor and destructor) as structures. Use
    classes only if functions other than set/get will be needed.

  \item Definition of constructors and destructors (for both classes
    and structs) and class member functions should not be included in
    header files, unless the definitions are very small.

  \item Declare all global variables in global vars.h and define them
    in main.cc.

  \item More to be added.
\end{itemize}




% LocalWords:  Eg Accessor bool struct typedef const NUM cond func defn arg
% LocalWords:  sizeof ifndef endif Doxygen accessor destructors structs
